\documentclass{article}
\usepackage[textwidth=170mm, textheight=230mm, inner=20mm, top=5mm, bottom=20mm]{geometry}
\usepackage[normalem]{ulem}
\usepackage[utf8]{inputenc}
\usepackage{physics}
\usepackage{scrextend}
\KOMAoption{fontsize}{11pt}
\PassOptionsToPackage{defaults=hu-min}{magyar.ldf}
\usepackage[magyar]{babel}
\usepackage{amsmath, amsthm,amssymb,paralist,array, ellipsis, graphicx, float, relsize}

\begin{document}
\renewcommand{\labelitemi}{\textbullet}
\def\R{\mathbb{R}}
\def\N{\mathbb{N}}
\def\br{\\[0.1cm]}
\begin{center}
	{\Large\textbf{Modellek és algoritmusok B szakirány}}\\[0.2cm]
	Tételkimondások és definíciók a 2. ZH-ra
\end{center}
{\small A kidolgozást \textsc{Bauer Bence} készítette \textsc{Dr. Weisz Ferenc} előadásai alapján.}\\
{\footnotesize Külön köszönet \textsc{Korponai Károlynak} a jegyzetekért.}

\begin{enumerate}
	\item\textbf{KÉP és Lineáris DER}\br
	Az $x'+f\cdot x=g,\quad x(\tau)=\xi$ KÉP globálisan egyértelműen megoldható lineáris\br
	differenciálegyenlet rendszer.
	\item\textbf{Lineáris differenciálegyenlet rendszer}\br
	$I\subset\R$ nyílt intervallum, $A:I\rightarrow\R^{n\times n}$ folytonos és korlátos,
	azaz $A(t)=[a_{i,j}(t)]\quad i,j=1,...,n$\br
	és $a_{i,j}$ folytonos és korlátos, továbbá $b:I\rightarrow\R^n$ folytonos és korlátos.\br
	Ekkor az $x'=A\cdot x+b$ feladatot linerás DER-nek nevezzük.
	\item\textbf{KÉP globálisan egyértelműen megoldhatósága}\br
	Az $x'=A\cdot x+b, x(\tau)=\xi$ KÉP globálisan egyértelműen megoldható
	$\forall(\tau, \xi)$ esetén.
	\item\textbf{Homogén-inhomogén megoldás jelölése}\br
	Jelölje $M_h$ az $x'=A\cdot x$ homogén DER összes teljes megoldását\br
	jelölje $M_{ih}$ az $x'=A_i\cdot x+b$ inhomogén DER összes teljes megoldását
	\item\textbf{Homogén megoldás dimenziója}\br
	Tekintsük az $x'=A\cdot x$ lineáris DER-t.\br
	Ekkor $M_h\subset C^1(I, \R^n)=\{\varphi:I\rightarrow\R^n,\varphi\in C^1\}$ altér és
	$dim(M_h)=n$
	\item\textbf{Homogén megoldás bázisa}\br
	Az $M_h$ egy bázisát a DER alaprendszerének nevezzük.
	\item\textbf{Alapmátrix definíciója}\br
	Ha $\varphi^{(1)},...,\varphi^{(n)}$ alaprendszer, akkor $\Phi=
	(\varphi^{(1)},...,\varphi^{(n)})$ alapmátrix.
	\item\textbf{Alapmátrix és megoldhatóság kapcsolata}\br
	Ha $\Phi\text{ az }x'=A\cdot x$ alapmátrixa, akkor $M_h=\{\Phi\cdot c : c\in\R^n\}$,
	azaz $\forall$megoldás felírható\br
	$\sum\limits_{i=1}^n c_i\cdot\varphi^{(i)}$ alakban, ahol $c=
	\begin{bmatrix}
	c_1 \\ . \\ . \\ c_n
	\end{bmatrix}$
	\item\textbf{Alaprendszer tétel}\br
	Tekintsük az $x'=A\cdot x$ lin DER-t. Tfh. $A$ állandó mátrix és $A$-nak $\exists$ $n$
	db lineárisan független saját vektora $s_1,...,s_n$, a hozzájuk tartozó
	$\lambda_1,...,\lambda_n$ sajátértékekkel.\br
	Ekkor $\varphi^{(n)}_{(t)}=s_i\cdot e^{\lambda_i\cdot t}\quad i=1,..,n$ egy alaprendszer.
	\item\textbf{Homogén és inhomogén megoldás kapcsolata}\br
	Ha $\psi_0\in M_{ih}$, akkor $M_{ih}=\{\psi_0+M_h\}=\{\psi_0+\varphi:\varphi\in M_h\}$
	\item\textbf{Inhomogén megoldás felírása}
	\begin{itemize}
		\item Az $x'=A\cdot x+b$ összes megoldása: $\psi(t)=\Phi(t)\cdot c+\Phi(t)\cdot
		\int\limits_{\tau}^t \Phi(s)^{-1}b(s)ds$
		\item Az $x'=A\cdot x+b, x(\tau)=\xi$ KÉP globálisan egyértelmű megoldható:\br
		$\psi(t)=\Phi(t)\cdot\Phi(t)^{-1}\cdot\xi+\Phi(t)\cdot\int\limits_{\tau}^t
		\Phi(s)^{-1}b(s)ds$
	\end{itemize}
	\newpage
	\item\textbf{n-edrendű DE}\br
	$D\subset\R^{n+1}$ tartomány, $h:I\rightarrow\R$ folytonos függvény.\br
	Az $y^{(y)}(t)=b(t,y(t),y'(t),..,y^{(n-1)}(t))$ feladatot n-edrendű DE-nek nevezzük, ahol\br
	$y:I\rightarrow\R,y\in C^n,I\subset\R$ nyílt intervallum.
	\item\textbf{Átviteli elv (n-edrendű DE-re)}\br
	$\varphi$ kielégíti az $y^{(n)}=h\circ(id,y,...,y^{(n-1)})$ DE-t $\Leftrightarrow\psi$
	kielégíti az $x'=f\circ(id,x)$ DER-t.
	\item\textbf{Magasabb rendű KÉP}\br
	Az $y^{(n)}=h\circ(id,y,...,y^{(n-1)}), y(t)=\xi_1,...,y^{(n-1)}(t)=\xi_n$
	feladatot KÉP-nek nevezzük.
	\item\textbf{Peano és Picard-Lindelöf tételek n-edrendű DER-en}\br
	A DER-re tanult Peano és Picard-Lindelöf tételek most is igazak.
	\item\textbf{n-edrendű lineáris DE}\br
	$a_0,...,a_{n-1}, b:I\rightarrow\R$ folytonos függvény, $I\subset\R$ nyílt.\br
	Az $y^{(n)}+a_{n-1}\cdot y^{(n-1)}+...+a_1\cdot y'+a_0\cdot y=b$ feladatot
	n-edrendű lineráis DE-nek nevezzük.
	\item\textbf{Átviteli elv (n-edrendű lineáris DE-re)}\br
	$\varphi$ kielégíti az $y^{(n)}+a_{n-1}\cdot y^{(n-1)}+...+a_1\cdot y'+a_0\cdot y=b$
	lin. DE-t $\Leftrightarrow\psi$ kielégíti az\br$x'=A\cdot x+\bar{b}$ lin. DER-t.
	\item\textbf{Homogén, inhomogén n-edrendű lineáris DE definíciója}\br
	$y^{(n)}+a_{n-1}\cdot y^{(n-1)}+...+a_1\cdot y'+a_0\cdot y = b$ lineáris DE
	homogén, ha $b=0$, inhomogén különben.
	\item\textbf{n-edrendű lineáris DE folytonossága és dimenziója}\br
	Tekintsük: $y^{(n)}+a_{n-1}\cdot y^{(n-1)}+...+a_0\cdot y=0$
	homogén DE-t.\br Ekkor $M_h\subset C^n (I, \R)$ altér és $\dim M_h=n$
	\item\textbf{Inhomogén megoldás felírása homogén megoldással}\br
	Ha $\psi_0\in M_{ih}$, akkor $M_{ih}=\psi_0+M_h$
	\item\textbf{DE karakterisztikus polinomja}\br
	A DE karakterisztikus polinomja: $K(z)=z^n+a_{n-1}\cdot z^{n-1}+...+a_1
	\cdot z+a_0$
	\item\textbf{Homogén megoldás és karakterisztikus polinom kapcsolata}\br
	$\varphi(t)=e^{\lambda\cdot t}\in M_h \Leftrightarrow\lambda$ gyöke K-nak
	\item\textbf{DE karakterisztikus polinomjának alaprendszere}\br
	Ha K-nak $\exists n$ db különböző $\lambda_1, ... , \lambda_n$ gyöke, akkor
	$\varphi_{j}(t)=e^{\lambda_j\cdot t},j=1,...,n$ alaprendszer.
	\item\textbf{$Valami_1$}\br
	Tfh. $K(z)=(z-\lambda_1)^{m_1}\cdot...\cdot(z-\lambda_r)^{m_r},\lambda_1,
	... ,\lambda_r$ különböző gyök, $m_1 + ... + m_r = n$.\br Ekkor
	$\varphi_{i,j}(t)=e^{\lambda_j\cdot t}\cdot t^i,j=1,..,r,i=0,..,m-1$
	alaprendszer.
	\item\textbf{Az inhomogén megoldása (?)}\br
	Ha $c(t)$ kielégíti a $\Phi(t)\cdot c'(t) = \bar{b}(t)$ egyenletet, akkor
	$c_1(t)\cdot\varphi_1(t)+...+c_n(t)\cdot\varphi_n(t)\in M_{ih}$
	\item\textbf{Az inhomogén megoldása, csak hosszaban lol (?)}\br
	Tekintsük az $y^{(n)}+a_{n-1}\cdot y^{(n-1)}+...+a_0\cdot y=b$ inhomogén
	DE-t.\br Legyen $\alpha,\beta,A,B,C,D\in\R$ és $\alpha\neq\beta_i$ k-szoros
	gyöke K-nak (ha nem gyök $\Rightarrow k=0$).\br Tfh.: $b(t)=p(t)\cdot
	e^{\lambda\cdot t}\cdot(A\cdot \cos(\beta\cdot t)+\beta\cdot\sin(\beta\cdot
	t)).$\br Ekkor $\exists\varphi(t)=t^k\cdot Q(t)\cdot e^{\alpha\cdot t}
	\cdot(C\cdot\cos(\beta\cdot t)+D\cdot\sin(\beta\cdot t)\in M_{ih}$,
	\br ahol P és Q polinomok és Q foka $\leq$ P foka.
	\newpage
	\item\textbf{Függvénysorozat, függvénysor definíciója}
	\begin{itemize}
		\item Az $(f_n)$ sorozatot függvénysorozatnak nevezzük.
		\item$\sum$ fv-t függvénysornak nevezzük, ahol $\sum fv_n=
		(\sum\limits_{k=0}^n f_k, n\in\N)$
	\end{itemize}
	\item\textbf{Függvénysorozat határértéke, függvénysor összege (?)}\br
	$KH(f_n)={x\in A:f_n(x)\text{ konvergens}}\br
	KH(\sum f_n)={x\in A:\sum f_n(x)\text{ konvergens}}\br
	\lim fn:KH(f_n)\rightarrow\R,x\mapsto lim f_n(x)\br
	\sum\limits_{n=0}^\infty f_n:KH(\sum f_n)\rightarrow\R,x\mapsto
	\sum\limits_{n=0}^\infty f_n(x)$
	\item\textbf{Függvénysorozat egyenletes konvergenciájának definíciója}\br
	Az $(f_n)$ fv.sorozat egyenletesen konvergens, ha\br$\exists f:A\rightarrow
	\R,\forall\varepsilon>0,\exists n_0\in\N,\forall n\geq n_0,\forall x\in A:
	\abs{f_n(x) - f(x)}< \varepsilon$\br
	Jele: %$f_n \{e nyíl} f$
	\item\textbf{$(f_n)$ egyenletes konvergenciája és $f_n,f$ kapcsolata}\br
	%f_n \{e nyíl} f%
	$\Rightarrow f_n\rightarrow f$ pontonként.
	\item\textbf{Függvénysorozat apszolút konvergenciája}\br
	$f_n$ egyenletesen konvergeny $\Leftrightarrow\forall\varepsilon>0,
	\exists n_0\in\N,\forall n,m\geq n_0,\forall x\in A:\abs{f_n(x) - f_m(x)}< \varepsilon$
	\item\textbf{Függvénysorozat határértékének folytonossága}\br
	Ha \begin{itemize}
		\item $f_n\in C(A)$
		\item $(f_n)$ egyenletesen konvergens
	\end{itemize}
	akkor $\lim f_n\in C(A)$
	\item\textbf{Weistras-tétel}\br
	Tekintsük a $\sum f_n$ függvénysort és a $\sum a_n$ számsort. Tfh:
	\begin{itemize}
		\item $\forall x\in A:\abs{fn(x)}\leq a_n$
		ii) $f_n$ egyenletesen konvergens.
	\end{itemize}
	Ekkor $f=\lim f_n\in R[a,b]$ és $\int\limits_a^b f=
	\lim\limits_{n->\infty}\int\limits_a^b f_n$
	\item\textbf{Függvénysor Reimann integrálhatósága}\br
	Tfh: \begin{itemize}
		\item $f_n:[a,b]\rightarrow\R,f_n\in R[a,b],n\in\N$
		\item $\sum f_n$ egyenletesen konvergens.
	\end{itemize}
	Ekkor $f=\sum\limits_{n=0}^\infty f_n\in R[a,b]$ és $\int\limits_a^b f=
	\sum\limits_{n=0}^\infty\int\limits_a^b f_n$
	\newpage
	\item\textbf{$Valami_2$}\br
	Tfh: \begin{itemize}
		\item $f_n(a,b)\rightarrow\R,f_n\in D(a,b), n\in\N$
		\item $\exists x_0\in(a,b):(f_n(x_0))$ konv.
		\item $(f_n')$ egyenletesen konvergens
	\end{itemize}
	Ekkor: \begin{itemize}
		\item $(f_n)$ egyenletesen konvergens
		\item $f=\lim f_n\in D(a,b)$
		\item $f'=\lim f_n'$
	\end{itemize}
\end{enumerate}
\end{document}