\documentclass{article}
\usepackage[textwidth=170mm, textheight=230mm, inner=20mm, top=5mm, bottom=20mm]{geometry}
\usepackage[normalem]{ulem}
\usepackage[utf8]{inputenc}
\usepackage{physics}
\usepackage{scrextend}
\KOMAoption{fontsize}{11pt}
\PassOptionsToPackage{defaults=hu-min}{magyar.ldf}
\usepackage[magyar]{babel}
\usepackage{amsmath, amsthm,amssymb,paralist,array, ellipsis, graphicx, float, relsize}

\begin{document}
\renewcommand{\labelitemi}{\textbullet}
\def\R{\mathbb{R}}
\def\br{\\[0.1cm]}
\begin{center}
	{\Large\textbf{Modellek és algoritmusok B szakirány}}\\[0.2cm]
	Tételkimondások és definíciók a 2. ZH-ra
\end{center}
{\small A kidolgozást \textsc{Bauer Bence} készítette \textsc{Dr. Weisz Ferenc} előadásai alapján.}\\
{\footnotesize Külön köszönet \textsc{Korponai Károlynak} és \textsc{Horváth Milánnak} a jegyzetekért.}

\begin{enumerate}
	\item
\end{enumerate}
\end{document}