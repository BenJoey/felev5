\documentclass{article}
\usepackage[textwidth=170mm, textheight=230mm, inner=20mm, top=5mm, bottom=20mm]{geometry}
\usepackage[normalem]{ulem}
\usepackage[utf8]{inputenc}
%\usepackage[T1]{fontenc}
\usepackage{physics}
\usepackage{scrextend}
\KOMAoption{fontsize}{11pt}
\PassOptionsToPackage{defaults=hu-min}{magyar.ldf}
\usepackage[magyar]{babel}
\usepackage{amsmath, amsthm,amssymb,paralist,array, ellipsis, graphicx, float, relsize}

\begin{document}
\renewcommand{\labelitemi}{\textbullet}
\begin{center}
	{\Large\textbf{Modellek és algoritmusok B szakirány}}\\[0.2cm]	
	Tételkimondások és definíciók az 1. ZH-ra
\end{center}
{\small A kidolgozást \textsc{Bauer Bence} készítette \textsc{Dr. Weisz Ferenc} előadásai alapján.}

\begin{enumerate}
	\item 
\end{enumerate}
\end{document}