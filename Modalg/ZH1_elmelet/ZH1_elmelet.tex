\documentclass{article}
\usepackage[textwidth=170mm, textheight=230mm, inner=20mm, top=5mm, bottom=20mm]{geometry}
\usepackage[normalem]{ulem}
\usepackage[utf8]{inputenc}
\usepackage{physics}
\usepackage{scrextend}
\KOMAoption{fontsize}{11pt}
\PassOptionsToPackage{defaults=hu-min}{magyar.ldf}
\usepackage[magyar]{babel}
\usepackage{amsmath, amsthm,amssymb,paralist,array, ellipsis, graphicx, float, relsize}

\begin{document}
\renewcommand{\labelitemi}{\textbullet}
\def\R{\mathbb{R}}
\def\br{\\[0.1cm]}
\def\folytdifh{folytonosan differenciálható }
\begin{center}
	{\Large\textbf{Modellek és algoritmusok B szakirány}}\\[0.2cm]
	Tételkimondások és definíciók az 1. ZH-ra
\end{center}
{\small A kidolgozást \textsc{Bauer Bence} készítette \textsc{Dr. Weisz Ferenc} előadásai alapján.}\\{\footnotesize Külön köszönet \textsc{Korponai Károlynak} és \textsc{Horváth Milánnak} a jegyzetekért}

\begin{enumerate}
	\item\textbf{Inverz függvény tétel:}\br
	Legyen $\Omega\subset\R^n$ nyílt halmaz és $f:\Omega\rightarrow\R^n$\br
	Tfh: $f$ \folytdifh és $\exists a\in\Omega:\text{ det}f'(a)\neq0$\br
	Ekkor: $\exists U\subset\Omega, V\subset\R^n$ nyílt halmazok, $a\in U, f(a)\in V,
	f|_U:U\rightarrow V$ bijektív,\br
	$(f|_U)^{-1}$ differenciálható és $(f^{-1})'(y)=[f'(f^{-1}(y))]^{-1},\quad (y\in V)$
	\item\textbf{Implicit függvény tétel} (speciális eset):\br
	Legyen $\Omega\subset\R^2$ nyílt, $f:\Omega\rightarrow\R$\br
	Tfh: $f$ folytonosan differenciálható $\Omega$-n és $\exists(a,b)\in\Omega:
	f(a,b)=0\text{ és }\delta_2f(a,b)\neq0$\br
	Ekkor:
	\begin{itemize}
		\item $\exists u_1, u_2\subset\R$ nyílt, $a\in u_1, b\in u_2\text{ és }\exists!\varphi:
		u_1\rightarrow u_2$ bijekció:$\varphi(a)=b\text{ és }f(x, \varphi(x))=0$
		\item $\varphi:u_1\rightarrow u_2$ \folytdifh és $\varphi'(x)=$
		\LARGE{$-\frac{\delta_1f(x, \varphi(x))}{\delta_2f(x, \varphi(x))}$}
	\end{itemize}
	Mindkét esetben $x\in u_1$
	\item\textbf{Implicit függvény tétel} (általános eset):\br
\end{enumerate}
\end{document}