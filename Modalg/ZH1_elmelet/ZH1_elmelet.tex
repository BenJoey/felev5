\documentclass{article}
\usepackage[textwidth=170mm, textheight=230mm, inner=20mm, top=5mm, bottom=20mm]{geometry}
\usepackage[normalem]{ulem}
\usepackage[utf8]{inputenc}
\usepackage{physics}
\usepackage{scrextend}
\KOMAoption{fontsize}{11pt}
\PassOptionsToPackage{defaults=hu-min}{magyar.ldf}
\usepackage[magyar]{babel}
\usepackage{amsmath, amsthm,amssymb,paralist,array, ellipsis, graphicx, float, relsize}

\begin{document}
\renewcommand{\labelitemi}{\textbullet}
\def\R{\mathbb{R}}
\def\br{\\[0.1cm]}
\def\folytdifh{folytonosan differenciálható }
\begin{center}
	{\Large\textbf{Modellek és algoritmusok B szakirány}}\\[0.2cm]
	Tételkimondások és definíciók az 1. ZH-ra
\end{center}
{\small A kidolgozást \textsc{Bauer Bence} készítette \textsc{Dr. Weisz Ferenc} előadásai alapján.}\\
{\footnotesize Külön köszönet \textsc{Korponai Károlynak} és \textsc{Horváth Milánnak} a jegyzetekért.}

\begin{enumerate}
	\item\textbf{Inverz függvény tétel:}\br
	Legyen $\Omega\subset\R^n$ nyílt halmaz és $f:\Omega\rightarrow\R^n$\br
	Tfh: $f$ \folytdifh és $\exists a\in\Omega:\text{ det}f'(a)\neq0$\br
	Ekkor: $\exists U\subset\Omega, V\subset\R^n$ nyílt halmazok, $a\in U, f(a)\in V,
	f|_U:U\rightarrow V$ bijektív,\br
	$(f|_U)^{-1}$ differenciálható és $(f^{-1})'(y)=[f'(f^{-1}(y))]^{-1},\quad (y\in V)$
	\item\textbf{Implicit függvény tétel} (speciális eset):\br
	Legyen $\Omega\subset\R^2$ nyílt, $f:\Omega\rightarrow\R$\br
	Tfh: $f$ folytonosan differenciálható $\Omega$-n és $\exists(a,b)\in\Omega:
	f(a,b)=0\text{ és }\delta_2f(a,b)\neq0$\br
	Ekkor:
	\begin{itemize}
		\item $\exists U_1, U_2\subset\R$ nyílt, $a\in U_1, b\in U_2\text{ és }\exists!\varphi:
		U_1\rightarrow U_2$ bijekció:$\varphi(a)=b\text{ és }f(x, \varphi(x))=0$
		\item $\varphi:U_1\rightarrow U_2$ \folytdifh és $\varphi'(x)=$
		\LARGE{$-\frac{\delta_1f(x, \varphi(x))}{\delta_2f(x, \varphi(x))}$}
	\end{itemize}
	Mindkét esetben $x\in U_1$
	\item\textbf{Implicit függvény tétel} (általános eset):\br
	Legyen $\Omega_1\subset\R^{n_1},\Omega_2\subset\R^{n_2}$ nyílt, 
	$f:\Omega_1\times\Omega_2\rightarrow\R^{n_2}$\br
	Tfh: $f$ \folytdifh $\Omega_1\times\Omega_2$-n és $\exists a\in\Omega_1, b\in\Omega_2:
	f(a,b)=0\br\text{és }det(\delta_2f(a,b))\neq0$\br
	Ekkor:
	\begin{itemize}
		\item $\exists U_1\subset\Omega_1, U_2\subset\Omega_2$ nyílt, $a\in U_1, b\in U_2\text{ és }\exists!\varphi:
		U_1\rightarrow U_2:\varphi(a)=b\text{ és }f(x, \varphi(x))=0$
		\item $\varphi:U_1\rightarrow U_2$ \folytdifh és $\varphi'(x)=$
		\Large{$-[\delta_2f(x,\varphi(x))]^{-1}\cdot\delta_1f(x,\varphi(x))$}
	\end{itemize}
	Mindkét esetben $x\in U_1$
	\item\textbf{Parciális derivált számítása implicit függvénnyel}
	\begin{itemize}
		\item $\delta_2f(a,b):=(y\mapsto f(a,y))'|_{y=b}\in\R^{n_2\times n_2}$
		\item $\delta_1f(a,b):=(x\mapsto f(x,b))'|_{x=a}\in\R^{n_2\times n_1}$
	\end{itemize}
	\item\textbf{Unknown def No.2}\br
	Az $f$ függvénynek a $c\in H$-ban feltételes lokális minimuma (maximuma) van, ha\br
	$\exists K(c):\forall x\in K(c)\cap H:f(x)\geq f(c),\quad(f(x)\leq f(c))$
	\item\textbf{Feltételes lokális szélsőérték szükséges feltétel:}\br
	Legyen $U\subset\R^n$ nyílt, $f,g_i:U\rightarrow\R,\quad i=1,...,m\quad$\br Tfh:
	\begin{itemize}
		\item $f,g_i$ \folytdifh\quad $i=1,...,m$
		\item $f$-nek feltételes lokális szélsőértéke van $c\in H$-ban
		\item $g'_i(c)$ vektorok lineárisan függetlenek
	\end{itemize}
	Ekkor: $\exists\lambda_1,...,\lambda_m\in\R$, hogy $L'(c)=0$, ahol
	$L=f+\lambda_1g_1+...+\lambda_mg_m$
	\newpage
	\item\textbf{Feltételes lokális szélsőérték elégséges feltétel:}\br
	Legyen $U\subset\R^n$ nyílt, $f,g_i:U\rightarrow\R,\quad i=1,...,m\quad$\br Tfh:
	\begin{itemize}
		\item $f,g_i$ \folytdifh\quad $i=1,...,m$
		\item $\exists\lambda_1,...,\lambda_m:L'(c)=0,\quad c\in H$
		\item $g_i(c)$ lineárisan független, $i=1,...,m$
		\item $L''(c)$ feltételesen pozitív definit
	\end{itemize}
	Ekkor $f$-nek $c$-ben feltételes lokális minimuma van.
	\item\textbf{Unknown def No.3}\br
	Feltételes pozitív definit, ha $A$ szimmetrikus mátrix $<A\cdot h,h>>0\quad
	\forall h\in\R^n\backslash\{0\}$\br amelyre $g'(c)\cdot h=0$, ahol $g=
	\begin{bmatrix}
		g_1 \\ . \\ . \\ g_m
	\end{bmatrix}$
	\item\textbf{Unknown def No.4}\br
	$\varphi:[\alpha,\beta]\rightarrow\R^n$ szakaszonként \folytdifh, ha $\varphi$
	folytonos és\br $\exists\alpha=t_0<t_1<...<t_n=\beta$, hogy
	{\Large $f|_{(t_i-1, t_i)}$} \folytdifh\quad $i=1,...,m-1$
	\item\textbf{Összefüggő halmaz}\br
	$U\subset\R^n$ összefüggő, ha $\forall x,y\in U:\exists\varphi:
	[\alpha,\beta]\rightarrow U$ szakaszonsan \folytdifh, $\varphi(\alpha)=x\text{ és }
	\varphi(\beta)=y$
	\item\textbf{Tartomány}\br
	$D\subset\R^n$ tartomány, ha nyílt és összefüggő.
	\item\textbf{Tartomány 2}\br
	$D\subset\R^{n+1}$ tartományon, ha $f:D\rightarrow\R^n$ folytonos
	\item\textbf{Elsőrendű explicit differenciál egyenlet:}\br
	Az $x'(t)=f(t,x(t))$ egyenletet elsőrendű explicit differenciál egyenletnek (DE) nevezzük, ahol\br
	$x:I\rightarrow\R^n$ \folytdifh, $I\subset\R$ nyílt intervallum.
	\item\textbf{Unknown def No.5}\br
	$D\subset\R^{n+1}, f:D\rightarrow\R^n$ folytonos, $(\tau, \xi)\in D,\quad
	\xi\in\R^n, \tau\in\R$
	\item\textbf{Kezdeti érték probléma}\br
	Az $x'(t)=f(t,x(t)),\quad x(t)=\xi$ feladatot kezdeti érték problémának (KÉP) nevezzük.
	\item\textbf{Peano tétel:}\br
	Ha $D\subset\R^{n+1}, f:D\rightarrow\R^n$ folytonos,  $(\tau, \xi)\in D$, akkor az\br
	$x'(t)=f(t,x(t)),\quad x(t)=\xi$ KÉP-nek $\exists$ megoldása.
	\item\textbf{KÉP. globális egyértelműen megoldhatósága}\br
	Az $x'(t)=f(t, x(t))\text{,  }x(\tau)=\xi$ KÉP globálisan, egyértelműen oldható meg, ha
	$\varphi$ és $\psi$ is megoldás, akkor $\varphi(t)=\psi(t)\text{,  }\forall t\in D_\varphi\cap D_\psi$
	\item\textbf{Teljes megoldás}\br
	Ha az $x'(t)=f(t, x(t))\text{,  }x(\tau)=\xi$ KÉP globálisan, egyértelműen oldható meg,
	akkor legyen\br$\tilde{\varphi}:=\bigcup\limits_{\varphi_{mo}}\varphi$, azaz
	$D_{\tilde{\varphi}}=\bigcup\limits_{\varphi_{mo}}D_\varphi\text{ és }
	\tilde{\varphi}(t)=\varphi(t)\quad t\in D_\varphi$\br
	$\tilde{\varphi}$-t teljes megoldásnak nevezzük.
	\newpage
	\item\textbf{KÉP. lokális egyértelmű megoldhatósága}\br
	Az $x'(t)=f(t, x(t))\text{,  }x(\tau)=\xi$ KÉP lokálisan, egyértelműen oldható meg, ha
	$\exists K(\tau,\xi)$ környezet, hogy erre leszűkítve $f$-et  a KÉP már globálisan egyértelmûen oldható meg.
	\item\textbf{KÉP Lokális és globális megoldhatóságának kapcsolata}\br
	Ha az $x'(t)=f(t, x(t))\text{,  }x(\tau)=\xi$ KÉP, $\forall(\tau,\xi)\in D$
	esetén lokálisan egyértelmûen oldható meg, akkor globálisan egyértelmûen megoldható.
	\item\textbf{Lokális Lipschitz feltétel}\br
	Az $f:D\rightarrow\R^n$ folytonos függvény kielégíti a második változójában a lokális Lipschitz
	feltételt a $(\tau,\xi)\in D$ pontban, ha $\exists K(\tau,\xi),\exists
	L:||f(t,u)-f(t,\tilde{u})||\subseteq L||u-\tilde{u}||\br\forall(t,u),(t,
	\tilde{u})\in K(\tau,\xi)\text{,  }t\in\R\text{  és  }u,\tilde{u}\in\R^n$
	\item\textbf{Folyt. diffhatóság és Lipschitz feltétel}\br
	Ha $f:D\rightarrow\R^n$ \folytdifh, akkor $f$ kielégíti a második változójában a lokális Lipschitz feltételt.
	$\quad\forall(\tau,\xi)\in D$-ben.
	\item\textbf{Picard-Lindelöf}\br
	Tfh: $f:D\rightarrow\R$ folytonos függvény kielégíti a második változóban a lokális Lipschitz feltételt a
	$(\tau,\xi)$ pontban. Ekkor: $x'(t)=f(t, x(t))\text{,  }x(\tau)=\xi$ KÉP lokálisan, egyértelműen oldható meg.
	\item\textbf{KÉP ekvivalens átalakítása}\br
	Az $x'(t)=f(t, x(t))\text{,  }x(\tau)=\xi$ KÉP ekvivalens az $x(t)=\xi+
	\int\limits_{\tau}^{t}f(s,x(s))ds$ integrálegyenlettel.
	\item\textbf{Lokális Lipschitz feltétel és KÉP globális megoldhatósága}\br
	Tfh: $f:D\rightarrow\R$ folytonos függvény kielégíti a második változóban a lokális Lipschitz feltételt a
	$(\tau,\xi)$ pontban. Ekkor: $x'(t)=f(t, x(t))\text{,  }x(\tau)=\xi$ KÉP globálisan, egyértelműen oldható meg.
	\item\textbf{Szeparális differenciál egyenlet}\br
	Ha $I_1,I_2\subset\R$ nyílt intervallumok, $f:I_1\rightarrow\R\text{,  }
	g:I_2\rightarrow\R$ folytonos függvények.\br
	Akkor az $x'(t)=f(t)\cdot g(x(t))$ feladatot szeparális differenciál egyenletnek nevezzük.
	\item\textbf{Szeparális DE globális megoldhatósága}\br
	Ha $0\notin R_g$, akkor az $x'=f\cdot g \circ x$ szeparális DE, az 
	$x(\tau)=\xi$ kezdeti állítás\br globálisan egyértelmûen oldható meg.
	\item\textbf{Lineárisan differenciál egyenlet}\br
	Legyen $I\subset\R$ nyílt intervallum, $f,g:\sum\rightarrow\R$ folytonos függvény.
	Az $x'(t)+f(t)\cdot x(t)=g(t)$ feladatot lineárisan differenciál egyenletnek nevezzük.
	\item\textbf{Lineárisan DE globális megoldhatósága}\br
	Az $x'+f\cdot x=g\text{,  }x(\tau)=\xi$ KÉP globálisan egyértelműen oldható meg.
	\item\textbf{Homogén-Inhomogén módszer}\br
	Legyen $\psi_0\in M_{\text{ih}}$. Ekkor $\psi\in M_{\text{ih}}\Leftrightarrow
	\exists\varphi\in M_{\text{h}}:\psi=\psi_0+\varphi$
\end{enumerate}
\end{document}