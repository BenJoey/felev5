\documentclass[12pt]{article}

\usepackage[a4paper,
inner = 15mm,
outer = 15mm,
top = 15mm,
bottom = 15mm]{geometry}

\usepackage{lmodern}
\usepackage[magyar]{babel}
\usepackage[utf8]{inputenc}
\usepackage[T1]{fontenc}
\usepackage[hidelinks]{hyperref}
\usepackage{graphicx}
\usepackage{amssymb}
\usepackage{xcolor}
\usepackage{epstopdf}
\usepackage{setspace}
\usepackage[nottoc,numbib]{tocbibind}
\usepackage{setspace}

%%Note: Azért van tele sortörésekkel mert triggereli az OCD-met ha egy egész bekezdés egy sorba van :D

\setstretch{1.2}
\begin{document}
\def\br{\\[0.1cm]}
\begin{titlepage}
	\vspace*{0cm}
	\centering
	\begin{tabular}{cp{1cm}c}
		\begin{minipage}{4cm}
			\vspace{0pt}
			\includegraphics[width=1\textwidth]{elte_cimer}
		\end{minipage} & &
		\begin{minipage}{7cm}
			\vspace{0pt}Eötvös Loránd Tudományegyetem \vspace{10pt} \newline
			Informatikai Kar \vspace{10pt} \newline
			Programozási Nyelvek és Fordítóprogramok Tanszék
		\end{minipage}
	\end{tabular}
	
	\vspace*{0.2cm}
	\rule{\textwidth}{1pt}
	
	\vspace*{3cm}
	{\Huge Osztott rendszerek specifikációja és implementációja }
	
	\vspace*{0.5cm}
	{\normalsize IP-08bORSIG}
	
	\vspace{2cm}
	{\huge Dokumentáció a 3. beadandóhoz}
	
	\vspace*{5cm}
	
	{\large \verb|Bauer Bence| }
	
	{\large \verb|JPSMA3| }
		
	
	\vfill
	
	\vspace*{1cm}
	2018. december 5.
\end{titlepage}
\thispagestyle{empty}
\begin{center}
	\colorbox{lightgray}{{\large JPSMA3} \hspace{4.3cm} {\large ORSI 3. beadandó} \hspace{5.7cm} \thepage}
\end{center}
\section{Feladat}

%%Csak a feladat "lényegi" részét másoltam ki a beadból
%%A magyarázós részeket, segítségeket és megkötéseket nem
%%De ha kellenek azok is akkor újra beküldöm és bemásolom az egészet :)

Az egyik bemeneti fájlban \textit{data.txt} \textbf{N} darab gyakornoki jelentkezéshez tartozó
információ olvasható, melyak alapján ki kell szűrni, hogy összesen hány kompetens jelentkező
van a rengeteg beérkezett pályázat között.\\[0.2cm]
\textit{data.txt} felépítése:

A fájl első sorában az \textbf{N} szám jelzi, hogy hány adatot olvasunk (ahol \textbf{N} egy nemnegatív egész szám).
A sorokban olvasható adatok függőleges vonal karakterrel (|) karakterrel vannak elválasztva egymástól.
Feltehetjük, hogy a | csak tagolási karakterként fordul elő egy-egy sorban, minden sor
helyesen van kitöltve, és pontosan annyi adat található a fájlban, amennyi az első sorban
olvasható, és nem található benne olyan sor, ahol kevesebb adat lenne megadva, vagy más módon
eltérne az elvárttól.

Adott egy másik fájl is \textit{filters.txt}, melyben összesen 7 különböző filter található.\\
\textit{filters.txt} felépítése:

A fájlban soronként egy filterhez tartozó adat olvasható. A három egész számú adatot
leszámítva a többiben | karakter szeparálja az információkat (pl. a képkiterjesztések sora
lehet: .png|.jpg|.jpeg) Feltehető, hogy minden sorban van értelmes és releváns adat (tehát
van elfogadott domain, meghirdetett állás, elvárt skill, elfogadott képformátum)

A beolvasott adatok alapján egy olyan adatcsatorna tételére visszavezetett megoldást kell
implementálni, mely beolvassa az inputadatokat, és az adatcsatorna egy-egy belső függvényének
egy-egy filtert megfeleltetve az inputadatokat végigfuttatja az összes szűrőn, majd kiírja az
\textit{output.txt} fájlba az adatokból azokat az e-mail címeket, akikhez tartozó
jelentkezésen az összes szűrő triggerelt.

\section{Felhasználói dokumentáció}
%%No idea what to write here :(

\subsection{Rendszer-követelmények, telepítés}

A programunk több platformon is futtatható, dinamikus függősége nincsen, bármelyik, manapság
használt PC-n működik. Külön telepíteni nem szükséges, elég a futtatható állományt elhelyezni
a számítógépen.

\subsection{A program használata}

A program használata egyszerű, külön paramétereket nem vár, így intézőből is indítható. A
futtatható állomány mellett kell elhelyezni az \textit{input.txt} valamint egy 
\textit{filters.txt} nevű fájlt, mely a bemeneti és a filterezési
adatokat tartalmazza, a fenti specifikációnak megfelelően. Figyeljünk az ebben található
adatok helyességére és megfelelő tagolására, mivel az alkalmazás külön ellenőrzést nem végez
erre vonatkozóan. A futás során az alkalmazás mellett található \textit{output.txt} fájl
tartalmazza a kapott eredményt, ami az összes filternek megfelelő jelentkezők email címei.

\newpage
\thispagestyle{empty}
\begin{center}
	\colorbox{lightgray}{{\large JPSMA3} \hspace{4.3cm} {\large ORSI 3. beadandó} \hspace{5.7cm} \thepage}
\end{center}

\section{Fejlesztői dokumentáció}

\subsection{Megoldási mód}
A feladatot az adatcsatorna tételére visszavezetve oldjuk meg.
\end{document}