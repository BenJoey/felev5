\documentclass[11pt,a4paper]{article}
\usepackage[utf8]{inputenc}
\usepackage[T1]{fontenc}
\usepackage{amsmath}
\usepackage[usenames,dvipsnames,svgnames,table]{xcolor}
\usepackage[normalem]{ulem}
\usepackage[left=1.00cm, right=1.00cm, top=0.40cm, bottom=1.0cm]{geometry}
\PassOptionsToPackage{defaults=hu-min}{magyar.ldf}
\usepackage[magyar]{babel}
\usepackage{framed, fancyhdr, wasysym, graphicx, multirow}

\begin{document}
\renewcommand{\labelitemi}{\textbullet}
\def\br{\\[0.1cm]}
\def\OldalKezdes{
\thispagestyle{empty}
\begin{center}
	\colorbox{lightgray}{{\large JPSMA3} \hspace{3cm} {\large Eseményvezérelt alkalmazások 2 1. beadandó} \hspace{5cm} \thepage}
\end{center}}
\OldalKezdes
\begin{framed}
	\begin{flushleft}
		{\large \textbf{Bauer Bence}}
		\hspace{3cm}{\large \textbf{1.Beadandó/19.Feladat}}
		\hspace{5.4cm}{\large 2018.10.06.}\br
		{\large JPSMA3}\br
		{\large bauerbence5@gmail.com}
	\end{flushleft}
\end{framed}
\section{Feladat}
Készítsünk programot, amellyel a potyogós amőba játékot lehet játszani, vagyis
az amőba azon változatát, ahol a jeleket felülről lefelé lehet beejteni a
játékmezőre. A játékmező itt is $nxn$-es tábla, és ugyanúgy X, illetve O jeleket
potyogtathatunk a mezőre. A játék akkor ér véget, ha betelik a tábla (döntetlen),
vagy valamelyik játékos kirak 4 egymás melletti jelet (vízszintesen, vagy átlósan).
A program minden lépésnél jelezze, hogy melyik játékos következik, és a tábla
egy üres mezőjére kattintva helyezhessük el a megfelelő jelet. Természetesen
csak a szabályos lépéseket engedje meg a program.
A program biztosítson lehetőséget új játék kezdésére a táblaméret megadásával
$(10 x 10, 20 x 20, 30 x 30)$, játék szüneteltetésére, valamint játék mentésére és
betöltésére. Ismerje fel, ha vége a játéknak, és jelenítse meg, melyik játékos
győzött (a táblán jelölje meg a győztes 4 karaktert). A program folyamatosan
jelezze külön-külön a két játékos gondolkodási idejét (azon idők összessége, ami
az előző játékos lépésétől a saját lépéséig tart, ezt is mentsük el és töltsük be).
\section{Elemzés}
\begin{itemize}
	\item A játékot tetszőleges méretű pályán játszhatjuk, melyet az ablak tetjén lévő
	mezővel tudunk beállítani, a "Generate Board" gomb lenyomásával új játék indul az
	éppen a mezőben lévő pályamérettel.
	\item A feladatot egyablakos asztali alkalmazásként Windows Forms grafikus felülettel
	valósítjuk meg.
	\item Az ablakon elhelyezünk egy File menüt (Mentés, Betöltés, Kilépés), Új játék
	beállítására szolgáló vezérlőt valamint egy státuszsort, mely a játékosok idejét és
	az éppen soron lévő játékost jelzi.
	\item A játéktáblát egy $nxn$-es megjelenítőkből álló rács és egy $n$ db nyomógombból
	álló sor reprezentálja. A egy nyomógombra való kattintás hatására a kattintott oszlop
	legalsó üres mezőjében megjelenik az éppen soron lévő játékos karaktere. Amint egy
	oszlop megtelt a nyomógombja inaktív lesz.
	\item A játék automatikusan feldob egy ablakot ha véget ért Döntetlen/Nyertes játékos
	nevével, valamint a két játékos idejével. Mentés/Betöltés esetén is dialógus ablakban
	történik a fájl kiválasztása.
	\item Felhasználói esetek:
\end{itemize}
\section{Tervezés}
\begin{itemize}
	\item Programszerkezet: A programot háromrétegű architektúrában valósítjuk meg:
	\begin{itemize}
		\item Megjelenítés: \textbf{View}
		\item Modell: \textbf{Model}
		\item Perzisztencia: \textbf{Persistence}
	\end{itemize}
	\item Perzisztencia:
	\begin{itemize}
		\item Feladata a mentés és betöltés biztosítása.
		\item Az adattárolást az \textbf{IAmobaDataAccess} interfész biztosítja, továbba
		lehetőséget ad a játék állapotának mentésére(\textbf{SaveAsync})/
		betöltésére(\textbf{LoadAsync})	melyeket aszinkron módon végzünk.
		\item A szöveges fájl kezelésére az \textbf{AmobaFileDataAccess} osztályt vezetjük be.
		Ha hiba lépett fel a\\fájlkezelés során akkor azt \textbf{AmobaDataException}-el jelezzük.
		\item A fájl első sorában eltároljuk a pálya méretét, a játékosok idejét, valamint
		a soron lévő játékost. A fájl többi része izomorf leképezése a játéktáblának, azaz
		$n$ sorban $n$ oszlop van szóközzel elválasztva, ahol rendre "X", "O", "E" jelöli
		az X játékos és az O játékos mezőit valamint az üres mezőket.
	\end{itemize}
\end{itemize}
\end{document}